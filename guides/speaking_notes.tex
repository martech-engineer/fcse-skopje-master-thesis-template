\documentclass[11pt, a4paper]{article}

% Packages for formatting
\usepackage[utf8]{inputenc}
\usepackage[T1]{fontenc}
\usepackage{geometry}
\usepackage{hyperref}
\usepackage{enumitem}
\usepackage{xcolor}
\usepackage{titlesec}

% Margins
\geometry{top=2.5cm, bottom=2.5cm, left=2.5cm, right=2.5cm}

% Hyperlink setup
\hypersetup{
    colorlinks=true,
    linkcolor=blue,
    filecolor=magenta,      
    urlcolor=cyan,
    citecolor=red,
}

% Title Info
\title{\textbf{Notes on "How To Speak"}}
\author{Based on the lecture by Prof. Patrick Winston}
\date{}

\begin{document}

\maketitle

\begin{abstract}
    This document contains structured notes derived from Steven Maude's summary of Professor Patrick Winston's famous talk "How to Speak" at MIT (2018). It covers strategies for starting, conveying ideas, using tools, and concluding effective presentations.
\end{abstract}

\tableofcontents
\vspace{1cm}

\section{Introduction: How to Start}
The beginning of a talk is critical for setting the tone and grabbing the audience's attention.
\begin{itemize}
    \item \textbf{Do not start with a joke.} The audience is not yet attuned to your speaking style, and jokes often fall flat at the very beginning.
    \item \textbf{Start with an Empowerment Promise.} Explicitly tell the audience what they will know at the end of the talk that they didn't know at the beginning.
    \item \textbf{No Laptops or Phones.} Humans have only one language processor. If it is engaged in reading email or browsing, they cannot listen. It also distracts neighbors and the speaker.
\end{itemize}

\section{Conveying Ideas}
To ensure your ideas are understood and remembered:
\begin{itemize}
    \item \textbf{Cycle on ideas.} Repeat key concepts multiple times to give people a chance to "get on the bus."
    \item \textbf{Fence your ideas.} Clearly distinguish your idea from others. Explicitly state how your approach differs from existing ones.
    \item \textbf{Verbal Punctuation.} Use landmarks (e.g., "First," "Second," "Now we move to...") to help the audience follow the structure and re-engage if they zoned out.
    \item \textbf{Ask Questions.} Ask questions that are neither too simple (insulting) nor too difficult (silence). Wait about 7 seconds for an answer.
\end{itemize}

\section{Time and Place}
\begin{itemize}
    \item \textbf{Time:} 11:00 AM is ideal. People are awake but not yet tired from lunch.
    \item \textbf{Lighting:} The room must be well-lit. Dim lights induce sleep. Do not turn off lights for slides; it is hard to see through closed eyelids.
    \item \textbf{Casing the Joint:} Visit the room beforehand to check for surprises.
    \item \textbf{Population:} The room should be appropriately sized. It is better to have a packed small room than a half-empty large hall.
\end{itemize}

\section{Tools: Boards, Props, and Slides}

\subsection{Boards (Chalk/Whiteboard)}
Boards are excellent for \textit{teaching} and \textit{informing}.
\begin{itemize}
    \item \textbf{Graphic Quality:} Allows for easy drawing and diagrams.
    \item \textbf{Pacing:} The speed of writing matches the audience's ability to absorb information.
    \item \textbf{Target for Hands:} Gives the speaker something to do with their hands, avoiding awkward pockets or behind-the-back postures.
\end{itemize}

\subsection{Props}
Props are memorable and effective due to \textbf{empathetic mirroring}. The audience can "feel" the weight or action of the object.

\subsection{Slides}
Slides are for \textit{exposing} ideas, not teaching them. Common crimes include:
\begin{itemize}
    \item \textbf{Reading the slides.} Never do this. The audience reads faster than you speak.
    \item \textbf{Laser Pointers.} Do not use them. They break eye contact. Use arrows on the slide instead.
    \item \textbf{Clutter.} Remove backgrounds, logos, and excessive bullets.
    \item \textbf{Font Size.} Use large fonts (40-50 pt). If you can't fit the text, you have too much text.
    \item \textbf{Hapax Legomenon.} You are allowed exactly \textit{one} overwhelmingly complex slide in a presentation to illustrate complexity, but no more.
\end{itemize}

\section{Persuading and Inspiring}

\subsection{Job Talks}
A job talk must demonstrate two things within the first 5 minutes:
\begin{enumerate}
    \item \textbf{Vision:} A problem someone cares about and a novel approach.
    \item \textbf{Execution:} That you have actually done something (list the steps).
\end{enumerate}

\subsection{Getting "Famous" (The Winston Star)}
To ensure your work is recognized, it should have:
\begin{itemize}
    \item \textbf{Symbol:} A visual representation (e.g., an arch).
    \item \textbf{Slogan:} A catchy phrase (e.g., "One-shot learning").
    \item \textbf{Surprise:} Something unexpected.
    \item \textbf{Salient Idea:} One key concept that sticks out.
    \item \textbf{Story:} A narrative of how it works.
\end{itemize}

\section{How to Stop}
The ending is the last impression you leave.
\begin{itemize}
    \item \textbf{Final Slide:} Do \textbf{not} use a "Collaborators" slide (put it first), a "Questions?" slide, or a "Thank You" slide.
    \item \textbf{The Contribution Slide:} End with a slide summarizing your contributions. This should stay up during Q\&A.
    \item \textbf{Final Words:} Do not end with a weak "Thank you." End with a \textbf{Salute} to the audience (e.g., "I value the time you've spent...") or a joke.
\end{itemize}

\section*{References}
\begin{enumerate}
    \item Steven Maude. \textit{Notes on "How To Speak" by Prof. Patrick Winston}. GitHub Gist. Available at: \url{https://gist.github.com/StevenMaude/280eadc60938ce4b6960dc60e830662d}
    \item MIT OpenCourseWare. \textit{How to Speak} (Prof. Patrick Winston). YouTube, 2019. \url{https://www.youtube.com/watch?v=Unzc731iCUY&t=315s} \cite{winston2019speak}
\end{enumerate}

% If you are using a .bib file, paste the following content into it:
% @misc{winston2019speak,
%   author = {Patrick Winston},
%   title = {How to Speak},
%   year = {2019},
%   publisher = {MIT OpenCourseWare},
%   howpublished = {YouTube},
%   url = {https://www.youtube.com/watch?v=Unzc731iCUY}
% }

\end{document}