\documentclass[a4paper,11pt]{article}

% Packages for functionality described in the guidelines
\usepackage{fontspec}           % Required for XeLaTeX font handling
\usepackage{graphicx}           % For including images
\usepackage{booktabs}           % For professional tables (\toprule, \midrule, \bottomrule)
\usepackage{amsmath}            % For advanced math and equations
\usepackage{amssymb}            % REQUIRED for \square symbol
\usepackage{hyperref}           % For hyperlinks and \autoref
\usepackage[noabbrev]{cleveref} % For intelligent referencing (\cref)
\usepackage{listings}           % For code listings
\usepackage{xcolor}             % For coloring code listings
\usepackage{geometry}           % For margin settings
\usepackage{enumitem}           % For customized lists

% Geometry settings
\geometry{margin=2.5cm}

% Code listing style
\lstset{
    basicstyle=\ttfamily\small,
    breaklines=true,
    frame=single,
    backgroundcolor=\color{gray!10},
    keywordstyle=\color{blue},
    commentstyle=\color{green!60!black},
    stringstyle=\color{purple},
    showstringspaces=false,
    language=[LaTeX]TeX
}

% Title Information
\title{\textbf{Quality Guidelines for References in LaTeX Overleaf}}
\author{}
\date{}

\begin{document}

\maketitle

\section{General Reference Principles}

\subsection{Label Naming Conventions}
Use descriptive, consistent prefixes to maintain clarity and organization:
\begin{itemize}
    \item \texttt{fig:} for figures
    \item \texttt{tab:} for tables
    \item \texttt{eq:} for equations
    \item \texttt{sec:} for sections
    \item \texttt{ch:} for chapters
    \item \texttt{alg:} for algorithms
    \item \texttt{lst:} for code listings
\end{itemize}

\subsection{Best Practices}
\begin{itemize}
    \item Always use \texttt{\textbackslash label\{\}} immediately after the caption.
    \item Reference using \texttt{\textbackslash ref\{\}}, \texttt{\textbackslash autoref\{\}}, or \texttt{\textbackslash cref\{\}}.
    \item \textbf{Never hardcode numbers} (e.g., write "Figure 1" manually—always use references).
    \item Place labels on the same line or immediately after captions.
\end{itemize}

\section{Figures and Images}

\subsection{Adding Figures}
Use the following structure to add figures:
\begin{lstlisting}
\begin{figure}[htbp]
    \centering
    \includegraphics[width=0.8\textwidth]{images/my-diagram.png}
    \caption{Description of the figure explaining what it shows}
    \label{fig:my-diagram}
\end{figure}
\end{lstlisting}

\subsection{Referencing Figures}
Different ways to reference figures:
\begin{lstlisting}
% Basic reference
As shown in Figure~\ref{fig:my-diagram}, the results indicate...

% With autoref (requires \usepackage{hyperref})
\autoref{fig:my-diagram} demonstrates the relationship between...

% With cleveref (requires \usepackage{cleveref})
\cref{fig:my-diagram} shows the data visualization...
\end{lstlisting}

\subsection{Figure Quality Guidelines}
\begin{itemize}
    \item \textbf{File formats:} Use PDF for vector graphics, PNG for screenshots; avoid JPEG for diagrams.
    \item \textbf{Resolution:} Minimum 300 DPI for print, 150 DPI for digital.
    \item \textbf{Sizing:} Use relative widths (e.g., \texttt{0.8\textbackslash textwidth}) rather than absolute measurements.
    \item \textbf{Placement:} Use \texttt{[htbp]} options (here, top, bottom, page).
    \item \textbf{Captions:} Should be descriptive and self-contained.
\end{itemize}

\section{Tables}

\subsection{Creating Tables}
Use the \texttt{booktabs} package for professional formatting:
\begin{lstlisting}
\begin{table}[htbp]
    \centering
    \caption{Performance comparison of different algorithms}
    \label{tab:performance-comparison}
    \begin{tabular}{lccr}
        \toprule
        Algorithm & Accuracy (\%) & Time (s) & Memory (MB) \\
        \midrule
        Method A & 85.2 & 12.4 & 256 \\
        Method B & 92.1 & 18.7 & 312 \\
        Method C & 88.9 & 9.2 & 198 \\
        \bottomrule
    \end{tabular}
\end{table}
\end{lstlisting}

\subsection{Referencing Tables}
\begin{itemize}
    \item The results presented in Table~\ref{tab:performance-comparison} show...
    \item \autoref{tab:performance-comparison} contains the experimental data...
\end{itemize}

\subsection{Table Quality Guidelines}
\begin{itemize}
    \item \textbf{Packages:} Use \texttt{booktabs} for professional-looking tables.
    \item \textbf{Alignment:} Left for text, center for short entries, right for numbers.
    \item \textbf{Captions:} Place above the table (convention).
    \item \textbf{Lines:} Use \texttt{\textbackslash toprule}, \texttt{\textbackslash midrule}, \texttt{\textbackslash bottomrule} instead of \texttt{\textbackslash hline}.
    \item \textbf{Formatting:} Keep it simple and readable.
\end{itemize}

\section{Equations and Formulas}

\subsection{Numbered Equations}
Only number equations you intend to reference.
\begin{lstlisting}
The PageRank algorithm can be expressed as:
\begin{equation}
    PR(A) = (1-d) + d \sum_{i=1}^{N} \frac{PR(T_i)}{C(T_i)}
    \label{eq:pagerank}
\end{equation}
\end{lstlisting}

\subsection{Referencing Equations}
\begin{itemize}
    \item According to Equation~\ref{eq:pagerank}, the PageRank value...
    \item Using \autoref{eq:pagerank}, we can calculate...
\end{itemize}

\subsection{Equation Guidelines}
\begin{itemize}
    \item \textbf{Numbering:} Only number equations you reference.
    \item \textbf{Unnumbered:} Use \texttt{equation*} or \texttt{\textbackslash[ ... \textbackslash]} for display math without numbers.
    \item \textbf{Inline:} Use \texttt{\$ ... \$} for inline mathematical expressions.
    \item \textbf{Punctuation:} Treat equations as part of sentences with proper punctuation.
\end{itemize}

\section{Sections and Cross-References}

\subsection{Section References}
\begin{lstlisting}
\section{Methodology}
\label{sec:methodology}

% Later in document
As described in Section~\ref{sec:methodology}...
The approach outlined in \autoref{sec:methodology} shows...
\end{lstlisting}

\subsection{Subsection References}
\begin{lstlisting}
\subsection{Data Collection}
\label{sec:data-collection}

% Reference
The process detailed in \autoref{sec:data-collection} involves...
\end{lstlisting}

\section{Overleaf-Specific Guidelines}

\subsection{File Organization}
Recommended project structure:
\begin{itemize}
    \item \texttt{project/}
    \begin{itemize}
        \item \texttt{main.tex}
        \item \texttt{references.bib}
        \item \texttt{chapters/}
        \begin{itemize}
            \item \texttt{introduction.tex}
            \item \texttt{methodology.tex}
            \item \texttt{conclusion.tex}
        \end{itemize}
        \item \texttt{images/}
        \begin{itemize}
            \item \texttt{diagram1.pdf}
            \item \texttt{chart2.png}
            \item \texttt{screenshot3.png}
        \end{itemize}
    \end{itemize}
\end{itemize}

\subsection{Overleaf Settings}
\begin{itemize}
    \item \textbf{Compiler:} Use XeLaTeX for Unicode support.
    \item \textbf{Main document:} Set correctly in project settings.
    \item \textbf{Auto-compile:} Enable for real-time preview.
    \item \textbf{Error handling:} Check logs for undefined references.
\end{itemize}

\section{Common Mistakes to Avoid}

\begin{table}[htbp]
    \centering
    \caption{Bad vs. Good Referencing Practices}
    \begin{tabular}{p{0.45\textwidth} p{0.45\textwidth}}
        \toprule
        \textbf{Don't Do This} & \textbf{Do This Instead} \\
        \midrule
        See Figure 1 for details... & See \autoref{fig:experimental-setup} for details... \\
        Table 2 shows the results... & \cref{tab:performance-results} shows the results... \\
        As mentioned in equation (3.1)... & As shown in \autoref{eq:optimization-function}... \\
        \texttt{\textbackslash label\{fig1\}} & \texttt{\textbackslash label\{fig:experimental-setup\}} \\
        \texttt{\textbackslash label\{table\}} & \texttt{\textbackslash label\{tab:performance-results\}} \\
        Figure \texttt{\textbackslash ref\{fig:example\}} & Figure\textasciitilde\texttt{\textbackslash ref\{fig:example\}} (use non-breaking space) \\
        \bottomrule
    \end{tabular}
\end{table}

\section{Quality Checklist}
Before submitting your document, ensure:
\begin{itemize}
    \item[$\square$] All figures have descriptive captions and labels.
    \item[$\square$] All tables use professional formatting (\texttt{booktabs}).
    \item[$\square$] All equations referenced in text are numbered and labeled.
    \item[$\square$] All citations appear in the bibliography.
    \item[$\square$] No hardcoded reference numbers.
    \item[$\square$] Consistent label naming convention used.
    \item[$\square$] Non-breaking spaces used before references (\texttt{\textasciitilde}).
    \item[$\square$] Document compiles without undefined reference warnings.
    \item[$\square$] All images are high quality and properly sized.
    \item[$\square$] Bibliography style is consistent and complete.
\end{itemize}

\end{document}