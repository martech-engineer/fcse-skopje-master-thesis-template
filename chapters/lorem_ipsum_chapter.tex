% ===================================================================
% CHAPTER 1: INTRODUCTION
% ===================================================================
\chapter{Вовед} % Introduction
\label{chap:intro}

% STRUCTURE: General Introduction
% Guidance: Introduce the broad field, the current landscape, and the specific domain of interest.
\noindent [General Field Context]...

\section{Мотивација и визија} % Motivation and Vision
\label{sec:motivation-vision}
% GUIDANCE:
% 1. The "Why": What is the gap in current solutions or literature?
% 2. The "Vision": What is the high-level ideal outcome of this work?

\section{Структура на трудот} % Structure of the Thesis
\label{sec:structure}
% GUIDANCE: Standard paragraph outlining the content of subsequent chapters (2-10).

% ===================================================================
% CHAPTER 2: PROBLEM DEFINITION
% ===================================================================
\chapter{Формулација на проблемот} % Problem Formulation
\label{chap:problem}

% GUIDANCE: Define the formal task. Use mathematical notation if applicable (Given X, find Y).
\noindent [Formal Problem Definition]...

\section{Цели и истражувачки прашања} % Goals and Research Questions
\label{sec:goals_qs}

% GUIDANCE:
% 1. Main Goal: One sentence statement of the primary objective.
% 2. RQs: Bulleted list of specific questions this thesis answers.
\begin{itemize}
    \item \textbf{RQ1:} [Metric/Performance Question]
    \item \textbf{RQ2:} [Comparative/Baseline Question]
    \item \textbf{RQ3:} [Impact/Parameter Question]
\end{itemize}

% ===================================================================
% CHAPTER 3: RELATED WORK
% ===================================================================
\chapter{Поврзани истражувања и литература} % Related Work
\label{chap:related_work}

% GUIDANCE: Divide literature into 3-4 logical pillars relevant to your topic.
% The source document [cite: 160] structure suggests:
% 1. Evolution of the field.
% 2. Theoretical foundations (the math/science behind it).
% 3. Current state-of-the-art approaches.

\section{Историски контекст и еволуција на [Field]} % Historical Context
\label{sec:history}

\section{Теоретски основи на [Core Technology]} % Theoretical Foundations
\label{sec:theory}

\section{Преглед на сродни методологии} % Overview of Related Methodologies
\label{sec:related_methods}

\section{Потребата за [Your Solution]} % The Need for Standardization/Automation
\label{sec:gap_analysis}
% GUIDANCE: Conclude by highlighting the specific gap your thesis fills.

% ===================================================================
% CHAPTER 4: METHODOLOGY
% ===================================================================
\chapter{Методологија} % Methodology
\label{chap:methodology}

% GUIDANCE: The "How" chapter. Based on, this follows a pipeline logic.

\section{Податоци} % Data
\label{sec:data}
% GUIDANCE: Describe dataset origin, size, cleaning, and preprocessing.

\section{[Core Algorithm/Model Description]} % e.g., "Deep Learning Model"
\label{sec:core_algorithm}
% GUIDANCE: Explain the main technical engine (Math/Logic) used.

\section{Дефинирање на експериментални услови} % Defining Experimental Conditions
\label{sec:exp_conditions}
% GUIDANCE: Instead of "Strategies", define the variations or parameters you are testing.

\section{Рамка за евалуација} % Evaluation Framework
\label{sec:eval_framework}
% GUIDANCE: Define the metrics used to measure success (Precision, Recall, Latency, ROI).

% ===================================================================
% CHAPTER 5: ARCHITECTURE
% ===================================================================
\chapter{Архитектурна логика на системот} % System Architectural Logic
\label{chap:architecture}

% GUIDANCE: Focus on the engineering/implementation details (Source: ).

\section{Користење на [Data Sources]} % Usage of Data Sources
\label{sec:data_sources}

\section{Детален системски дизајн и податочен проток} % Detailed Design & Data Flow
\label{sec:system_design}
% GUIDANCE: Include a flowchart figure here.

\section{Процесирање и трансформација на податоци} % Data Processing & Transformation
\label{sec:data_proc}
% GUIDANCE: How raw data becomes usable features (Vectorization, Normalization, etc.).

\section{Развојна околина и технологии} % Development Environment
\label{sec:dev_env}
% GUIDANCE: Libraries, Languages, Infrastructure choices.

% ===================================================================
% CHAPTER 6: USER INTERFACE (Optional/If Applied)
% ===================================================================
\chapter{Главни кориснички интерфејс компоненти} % Main UI Components
\label{chap:ui}

% GUIDANCE: Screenshots and descriptions of the tool built (if applicable).
% If no UI, rename to "System Modules" or "API Specifications".

% ===================================================================
% CHAPTER 7: EXPERIMENTS AND RESULTS
% ===================================================================
\chapter{Експерименти и резултати} % Experiments and Results
\label{chap:results}

% GUIDANCE: Based on[cite: 161, 162], this chapter is heavily structured.

\section{Експериментална поставеност} % Experimental Setup
\label{sec:exp_setup}
% GUIDANCE: Hardware, Test Data vs Validation Data, Control Variables.

\section{Користени метрики} % Metrics Used
\label{sec:metrics_used}
% GUIDANCE: Brief recap of primary and derived metrics.

\section{Анализа на примарни резултати} % Analysis of Primary Results
\label{sec:primary_results}
% GUIDANCE: The main performance data. (Replaces "Effectiveness of Strategies").

\section{Анализа на варијабилност и параметри} % Analysis of Variability/Parameters
\label{sec:parameter_analysis}
% GUIDANCE: How changing input X affects output Y. (Replaces "Topology/Connection Range").

\section{Компаративна анализа} % Comparative Analysis
\label{sec:comparative_analysis}
% GUIDANCE: Comparing Method A vs Method B (or Algorithm vs Baseline).
% (Replaces "Human vs Automated").

\section{Импликации за практична примена} % Implications for Practice
\label{sec:implications}

% ===================================================================
% CHAPTER 8: LIMITATIONS
% ===================================================================
\chapter{Ограничувања} % Limitations
\label{chap:limitations}

% GUIDANCE: Honest assessment of scope constraints (e.g., Data availability, Compute power).

% ===================================================================
% CHAPTER 9: CONCLUSION
% ===================================================================
\chapter{Заклучок и дискусија} % Conclusion and Discussion
\label{chap:conclusion}

% GUIDANCE: Summary of findings relative to the RQs defined in Chapter 2.

% ===================================================================
% CHAPTER 10: FUTURE WORK
% ===================================================================
\chapter{Идни планови и развојна патека} % Future Plans and Roadmap
\label{chap:future}

% GUIDANCE: Timeline or logic for future improvements (Source: [cite: 162]).